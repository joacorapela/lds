\documentclass[fleqn]{article}

\usepackage{natbib}
\usepackage{apalike}
\usepackage[hypertexnames=false,colorlinks=true,breaklinks]{hyperref}
\usepackage{amsmath,bm}
\usepackage{amssymb}

\makeatletter
\newenvironment{rowvector}
 {\bm{[}\begin{matrix}}
 {\end{matrix}\bm{]}}

\renewcommand*\env@matrix[1][\arraystretch]{%
  \edef\arraystretch{#1}%
  \hskip -\arraycolsep
  \let\@ifnextchar\new@ifnextchar
  \array{*\c@MaxMatrixCols c}}
\makeatother

\title{Discrete Wiener process acceleration model for tacking}
\author{Joaquin Rapela\thanks{j.rapela@ucl.ac.uk}}

\begin{document}

\maketitle

\abstract{
%
Starting with the equations of motion of a single particle, here I derive the
discrete Wiener process acceleration (DWPA) model~\citep[][Section 6.3.3; a
linear dynamical system for tracking an object moving in one
dimension]{barShalomEtAl2004}. Then I extend this model for tracking an object
moving in two dimensions (Section~\ref{sec:dwpaModel2D}).
%
}

\section{DWPA model in one dimension}
\label{sec:dwpaModel1D}

The DWPA model for tracking the motion of an object in one dimension is a
linear dynamical system

\begin{alignat*}{3}
    \mathbf{x}_n &= A\mathbf{x}_{n-1} + \mathbf{w}_n && \quad \text{with} \quad\mathbf{w}_n\sim N(\mathbf{0}, Q)\\
    y_n &= C\mathbf{x}_{n} + v_n                     && \quad \text{with} \quad v_n\sim N(0, \sigma^2)
\end{alignat*}

\noindent where

\begin{itemize}

    \item[--]
        $\mathbf{x}_n=\left[\xi[n],\dot{\xi}[n],\ddot{\xi}[n]\right]^\intercal$
        and $\xi[n],\dot{\xi}[n],\ddot{\xi}[n]$ are random variables representing
        the position, velocity and acceleration of the object at sample time
        $n$,

    \item[--]
        \begin{equation}
            A=\begin{bmatrix}
                1 & T & T^2\\
                0 & 1 & T\\
                0 & 0 & 1
            \end{bmatrix}
            \label{eq:A}
        \end{equation}
        and $T$ is the sample period,

    \item[--]
        \begin{equation}
            Q=\gamma^2\begin{bmatrix}[1.5]
                \frac{1}{4}T^4&\frac{1}{2}T^3&\frac{1}{2}T^2\\
                \frac{1}{2}T^3&T^2&T\\
                \frac{1}{2}T^2&T&1
            \end{bmatrix}
            \label{eq:Q}
        \end{equation}

    \item[--]
        \begin{align*}
            C=\begin{bmatrix}[1,5]
                1 & 0 &0
            \end{bmatrix}
        \end{align*}
\end{itemize}

We next derive the forms of $A$ and $Q$ from the equations of motion of a
particle. Matrix $C$ simply extracts the position component from the state
vector $x_n$.

\subsection*{Derivation of the form of matrices $A$ and $Q$}

Consider the Taylor series expansion of the position as a function of time,
$\xi(t)$, up to second order (Eq.~\ref{eq:posTaylor2}). Approximations of the
velocity and acceleration are derived from Eq.~\ref{eq:posTaylor2} by
successive differentiation (with respect to $T$) in Eqs.~\ref{eq:velTaylor2}
and~\ref{eq:accTaylor2}.

\begin{align}
    \xi(t+T)&=\xi(t)+\dot{\xi}(t)T+\frac{\ddot{\xi}(t)}{2}T^2\label{eq:posTaylor2}\\
    \dot{\xi}(t+T)&=\dot{\xi}(t)+\ddot{\xi}(t)T\label{eq:velTaylor2}\\
    \ddot{\xi}(t+T)&=\ddot{\xi}(t)\label{eq:accTaylor2}
\end{align}

According to Eq.~\ref{eq:accTaylor2} the approximation of the acceleration,
$\ddot{\xi}(t)$ is constant across all times. The DWPA model generalises this
by assuming that accelerations are constant only during each sampling period
of length $T$, with value equal to the second derivative of the position at the
start of the sampling period (i.e., $\ddot{\xi}(kT)$) plus a random value
$v(k)\sim\mathcal{N}(0,\gamma^2)$ (Eq.~\ref{eq:accDiscreteApprox}).

\begin{align}
    \ddot{\xi}_a(t)=\ddot{\xi}(kT)+v(k)\quad t\in[kT,(k+1)T)\label{eq:accDiscreteApprox}
\end{align}

Replacing $\ddot{\xi}_a(t)$ by $\ddot{\xi}(t)$ in Eqs.~\ref{eq:posTaylor2},
\ref{eq:velTaylor2} and~\ref{eq:accTaylor2} and discretizing we obtain in
Eqs.~\ref{eq:posDWPA}, \ref{eq:velDWPA} and~\ref{eq:accDWPA} the motion
equations for the DWPA model.

\begin{align}
    \xi(k+1)&=\xi(k)+\dot{\xi}(k)T+\frac{\ddot{\xi}(k)}{2}T^2+\frac{v(k)}{2}T^2\label{eq:posDWPA}\\
    \dot{\xi}(k+1)&=\dot{\xi}(k)+\ddot{\xi}(k)T+v(k)T\label{eq:velDWPA}\\
    \ddot{\xi}(k+1)&=\ddot{\xi}(k)+v(k)\label{eq:accDWPA}
\end{align}

Calling $x(k)=[\xi(k), \dot{\xi}(k), \ddot{\xi}(k)]^\intercal$,
Eq.~\ref{eq:DWPAmodel} rewrites the previous
equations in matrix form.

\begin{align}
    x(k)&=\begin{bmatrix}
        1 & T & T^2\\
        0 & 1 & T\\
        0 & 0 & 1
           \end{bmatrix}
           x(k-1)+
           \begin{bmatrix}
               \frac{1}{2}T^2\\
               T\\
               1
           \end{bmatrix}
           v(k)\nonumber\\
           &=Ax(k-1)+\Gamma v(k)\nonumber\\
           &=Ax(k-1)+w(k)\label{eq:DWPAmodel}
\end{align}

\noindent with 

\begin{align}
    A&=\begin{bmatrix}
          1 & T & T^2\\
          0 & 1 & T\\
          0 & 0 & 1
    \end{bmatrix}\label{eq:A}\\
    \Gamma&=\begin{bmatrix}
               \frac{1}{2}T^2\\
               T\\
               1
            \end{bmatrix}\nonumber\\
    w(k)&=\Gamma v(k)\nonumber
\end{align}

Because $v(k)\sim\mathcal{N}(0,\gamma^2)$ then $w(k)$ is also Gaussian with
mean zero and covariance $Q$ (Eq.~\ref{eq:Q}).

\begin{align}
    E\{w(k)\}&=\Gamma E\{v(k)\}=0\nonumber{}\\
    E\{w(k)w(k)^\intercal\}&=\Gamma E\{v(k)^2\}\Gamma^\intercal=\Gamma\gamma^2\Gamma^\intercal=\gamma^2\Gamma\Gamma^\intercal\nonumber\\
     &=\gamma^2\begin{bmatrix}[1.5]
         \frac{1}{2}T^2\\
         T\\
         1
       \end{bmatrix}
     \raisebox{4.5ex}{
         $\begin{rowvector}\frac{1}{2}T^2, T, 1\end{rowvector}$
     }\nonumber\\
     &=\gamma^2\begin{bmatrix}[1.5]
                   \frac{1}{4}T^4&\frac{1}{2}T^3&\frac{1}{2}T^2\\
                   \frac{1}{2}T^3&T^2&T\\
                   \frac{1}{2}T^2&T&1
     \end{bmatrix}=Q\label{eq:Q}
\end{align}

\noindent with $Q=\gamma^2Q_e$ and

\begin{align}
    Q_e=\begin{bmatrix}[1.5]
             \frac{1}{4}T^4&\frac{1}{2}T^3&\frac{1}{2}T^2\\
             \frac{1}{2}T^3&T^2&T\\
             \frac{1}{2}T^2&T&1
        \end{bmatrix}\label{eq:Qe}
\end{align}

\section{DWPA model in two dimensions}
\label{sec:dwpaModel2D}

We extend the DWPA model from one to two dimensions considering each of the two
dimensions independent of each other. This independence yields the following
linear dynamical system.

\begin{alignat*}{3}
    \tilde{\mathbf{x}}_n &= \tilde{A}\mathbf{x}_{n-1} + \tilde{\mathbf{w}}_n && \quad \text{with} \quad\tilde{\mathbf{w}}_n\sim N(\mathbf{0}, \tilde{Q})\\
    \tilde{\mathbf{y}}_n &= \tilde{C}\tilde{\mathbf{x}}_{n} + \tilde{\mathbf{v}}_n && \quad \text{with} \quad \tilde{\mathbf{v}}_n\sim N(0, \tilde{R})
\end{alignat*}

\noindent where

\begin{itemize}

    \item[--]
        $\mathbf{x}_n=\left[\xi_1[n],\dot{\xi_1}[n],\ddot{\xi_1}[n],\xi_2[n],\dot{\xi_2}[n],\ddot{\xi_2}[n]\right]^\intercal$
        and $\xi_i[n],\dot{\xi_i}[n],\ddot{\xi_i}[n]$ are random variables representing
        the position, velocity and acceleration of the object along dimension
        $i$ at sample time
        $n$,

    \item[--]
        \begin{align*}
            \tilde{A}&=\begin{bmatrix}
                A & \mathbf{0}\\
                \mathbf{0} & A\\
            \end{bmatrix}\nonumber\\
        \end{align*}
        and $A$ is given in Eq.~\ref{eq:A}

    \item[--]
        \begin{align*}
            \tilde{Q}=\begin{bmatrix}[1.5]
                \gamma^2_1Q_e & \mathbf{0}\\
                \mathbf{0} & \gamma^2_2Q_e
            \end{bmatrix}
        \end{align*}
        and $Q_e$ is given in Eq.~\ref{eq:Qe}

    \item[--]
        \begin{align*}
            \tilde{C}=\begin{bmatrix}[1.5]
                1 & 0 & 0 & 0 & 0 & 0\\
                0 & 0 & 0 & 1 & 0 & 0
            \end{bmatrix}
        \end{align*}

    \item[--]
        \begin{align*}
            \tilde{R}=\begin{bmatrix}[1.5]
                \sigma^2_1 & 0\\
                0 & \sigma^2_2
            \end{bmatrix}
        \end{align*}
\end{itemize}


\bibliographystyle{apalike}
\bibliography{tracking}

\end{document}
